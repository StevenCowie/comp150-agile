\documentclass{scrartcl}


\usepackage[hidelinks]{hyperref}
\usepackage[none]{hyphenat}

\title{Essay Proposal}
\subtitle{COMP110 - Computer Architecture Essay}

\author{Steven Cowie}

\begin{document}

\maketitle

\section*{Topic}

My essay will be on : What are the key challeneges game developers encounter when applying agile in a remote context whilst using SCRUM

% Add details as appropriate.

\section*{Paper 1}
% This is an example! Replace the details with a paper relevant to your chosen topic.
\begin{description}
\item[Title:] Back to Basics: The Role of Agile Principles in Success with an Distributed Scrum Team
\item[Citation:] \cite{4293626}
\item[Abstract:] ``Agile processes rely on feedback and communication to work and they often work best with co-located teams for this reason. Sometimes agile makes sense because of project requirements and a distributed team makes sense because of resource constraints. A distributed team can be productive and fun to work on if the team takes an active role in overcoming the barriers that distribution causes. This is the story of how one product team used creativity, communications tools, and basic good engineering practices to build a successful product.''
\item[Web link:] \url{http://ieeexplore.ieee.org.ezproxy.falmouth.ac.uk/document/4293626/}
\item[Full text link:] \url{http://ieeexplore.ieee.org.ezproxy.falmouth.ac.uk/xpls/icp.jsp?arnumber=4293626}
\item[Comments:] Talks about how scrum works efficiently if the team has an active role in overcoming the barriers that distribution causes, this is relevant because it talks about solutions to overcome the challenges they faced.
\end{description}

\section*{Paper 2}
\begin{description}
\item[Title:] Distributed Agile Development: Using Scrum in a Large Project
\item[Citation:] \cite{4638656}
\item[Abstract:] While seemingly incompatible, combining large-scale global software development and agile practices is a challenge undertaken by many companies. Case study reports on the successful use of agile practices in small distributed projects already exist. How these practices could be applied to larger projects, however, remains unstudied. This paper reports a case study on agile practices in a 40-person development organization distributed between Norway and Malaysia. Based on seven interviews in the development organization, we describe how Scrum practices were successfully applied, e.g., using teleconference and web cameras for daily scrum meetings, synchronized 4-week sprints and weekly scrum-of-scrums. Additional agility supporting practices for distributed projects were identified, e.g., frequent visits, unofficial distributed meetings and annual gatherings are described.
\item[Web link:] \url{http://ieeexplore.ieee.org.ezproxy.falmouth.ac.uk/document/4638656/}
\item[Full text link:] \url{http://ieeexplore.ieee.org.ezproxy.falmouth.ac.uk/xpls/icp.jsp?arnumber=4638656}
\item[Comments:] Speaks about how scrum helps break down the key challenge in distributed agile which is communication on a 40 man project between 2 teams in Norway and Malaysia. This is important because they narrow down their main probelm to be communication. 
\end{description}

\section*{Paper 3}
\begin{description}
\item[Title:] Why Scrum Works: A Case Study from an Agile Distributed Project in Denmark and India
\item[Citation:] \cite{6005502}
\item[Abstract:] Scrum seems to work extremely well as an agile project management approach. An obvious question is why. To answer that question, we carried out a longitudinal case study of a distributed project using Scrum across Denmark and India. In our analysis of case study data we used three selected theoretical frameworks. We conclude that Scrum works so well because it provides communication, social integration, control, and coordination mechanisms that are especially useful for distributed and agile project management.
\item[Web link:] \url{http://ieeexplore.ieee.org.ezproxy.falmouth.ac.uk/document/6005502/}
\item[Full text link:] \url{http://ieeexplore.ieee.org.ezproxy.falmouth.ac.uk/document/6005502/}
\item[Comments:] Talks about the challenges of distributed scrum, how they overcame it in their project and why they think it works so well because it provides communication, social integration, control, and coordination mechanisms if done correctly.
\end{description}

\section*{Paper 4}
\begin{description}
\item[Title:] Upgrading distributed agile development
\item[Citation:] \cite{6634027}
\item[Abstract:] Agile software development teams are normally collocated a matter that enhances the implementation of its principles. In the growing world of software industry it is better to get benefit of markets, skills and reduced cost which are distributed all over the world. This can be achieved by upgrading the practices of agile software development to be in harmony with distributed software development and keeping its intrinsic values from being destroyed or even distorted in spite of geographical barriers, different cultures and different time zones. This paper studies an approach based on sequential development of the software to help in dealing with the challenges that face agile methodology when it is combined with distributed software development.
\item[Web link:] http://ieeexplore.ieee.org.ezproxy.falmouth.ac.uk/document/6634027/
\item[Full text link:] http://ieeexplore.ieee.org.ezproxy.falmouth.ac.uk/xpls/icp.jsp?arnumber=6634027
\item[Comments:] Talks about the advantages of when the team is distributed. 
\end{description}

\section*{Paper 5}
\begin{description}
\item[Title:] Problems in the Adoption of Agile-Scrum Methodologies: A Systematic Literature Review
\item[Citation:] \cite{7477924}
\item[Abstract:] Agile methodologies are focused on the people and functional product delivery in short periods of time. There are methodologies that change considerably the work habits of software developers. Scrum is an agile methodology that involves an iterative, incremental, and empiric process. Besides it is designed to add value, focus, clarity and transparency to the activities and products of a project. Nowadays, most companies are interested in the adoption of agile methodologies. Although Scrum is a light process and easy to understand, its adoption sometimes is difficult. Agile methodologies are not obvious by themselves, so they are difficult to introduce in the culture of a company. In order to identify the problems presented during the adoption, a Systematic Literature Review is performed focusing in Scrum. We found several problems, these are categorized in four groups: people, process, project, and company (organization). The results represent a basis to propose a framework to support the agile adoption.
\item[Web link:] http://ieeexplore.ieee.org.ezproxy.falmouth.ac.uk/document/7477924/
\item[Full text link:] http://ieeexplore.ieee.org.ezproxy.falmouth.ac.uk/document/7477924/
\item[Comments:] Author talks about how scrum is used to add value, focus and clarity on task at hand.
\end{description}

\bibliographystyle{IEEEtran}
\bibliography{initial_references}

\end{document}
